% vim: set fo=tcqw :
\documentclass[12pt]{article}
\usepackage[latin1]{inputenc}
\usepackage{cmbright}
\usepackage[pdftex,margin=0pt,papersize={24cm,14cm},noheadfoot]{geometry}
\usepackage{tikz}
\usetikzlibrary{shapes,arrows,shadows,positioning,calc}

\colorlet{HEAD}{gray!50}
\colorlet{branch}{orange!50}
\colorlet{commit}{green!50}
\colorlet{index}{blue!40}
\colorlet{work}{red!40}
\colorlet{old HEAD}{HEAD!50}
\colorlet{old branch}{branch!50}

\tikzset{basic/.style = {rectangle, text width=5cm, text centered,
rounded corners, minimum height=1.5cm, line width=3pt}}

\tikzset{history/.style = {basic, fill=commit!40, draw=commit}}
\tikzset{index/.style = {basic, fill=index!50, draw=index}}
\tikzset{work/.style = {basic, fill=work!50, draw=work}}

\tikzset{commit/.style = {history, text width=2cm, minimum height=1cm}}
\tikzset{branch/.style = {commit, fill=branch!50, draw=branch}}
\tikzset{HEAD/.style = {commit, fill=HEAD!50, draw=HEAD}}
\tikzset{old branch/.style = {commit, fill=old branch!50, draw=old branch,
text=black!30}}
\tikzset{old HEAD/.style = {commit, fill=old HEAD!50, draw=old HEAD,
text=black!30}}
\tikzset{attached HEAD/.style = {HEAD, minimum height=2cm}}
\tikzset{attached HEAD label/.style = {basic, minimum height=1cm, text width=2cm}}
\tikzset{old attached HEAD/.style = {old HEAD, minimum height=2cm}}
\tikzset{old attached HEAD label/.style = {attached HEAD label, text=black!30}}

\tikzset{line/.style = {->, >=stealth', line width=1.5pt, color=black!70,
auto=right, bend right}}
\tikzset{object line/.style = {->, >=stealth, line width=1.5pt}}
\tikzset{commit line/.style = {object line, color=commit}}
\tikzset{HEAD line/.style = {object line, color=HEAD}}
\tikzset{branch line/.style = {object line, color=branch}}
\tikzset{old HEAD line/.style = {object line, color=old HEAD}}
\tikzset{old branch line/.style = {object line, color=old branch}}
\tikzset{cancel ref/.style = {cross out, draw=black!70, line width=1pt}}

\tikzset{background line/.style = {draw=white, line width=6pt,
    shorten >=6pt, shorten <=6pt}}

\tikzset{cmd/.style = {color=black, font={\Large\ttfamily}}}

\tikzset{main/.style = {node distance=.7cm, bend angle=35, font={\Large},
    x=1.5cm, y=1.5cm}}

\pgfdeclarelayer{HEAD}
\pgfsetlayers{HEAD,main}

\newcommand\boundingbox{\clip (-.5\textwidth,-.5\textheight) rectangle
    (.5\textwidth,.5\textheight);}

\newcommand\historynode[1][0,.5]{\node [history] (head) at (#1) {History};}
\newcommand\indexnode[1][0,-1]{\node [index] (index) at (#1) {Stage (Index)};}
\newcommand\worknode[1][0,-2.5]{\node [work] (work) at (#1) {Working Directory};}
\newcommand\normalcommits[1][6,.75]{
    \commitTerminal cT at (#1);
    \commit a47c3 (A) -> cT;
    \commit b325c (B) -> A;
    \commit c10b9 (C) -> B;
    \commit da985 (D) -> C;
    \commit ed489 (E) -> D;
}
\newcommand\usualsetup{
    \normalcommits
    \indexnode
    \worknode
}

\def\ref [#1] "#2" : #3 -> #4;{
    \node [#1, above=.4cm of #4] (#3) {#2};
    \path [#1 line] (#3) edge (#4);
    }
\def\oldref [#1] "#2" : #3 -> #4;{
    \ref [old #1] "#2" : old #3 -> #4;
    \node [cancel ref, above=.2cm of #4, anchor=center] {};
    }

\newcounter{tmp}
\def\commitTerminal #1 at (#2);{
    \node (#1) at (#2) {\LARGE$\cdots$};
    \pgfkeys{/child count/#1/.initial=0}
}
\def\commit #1 (#2) -> #3;{
    \setcounter{tmp}{\pgfkeysvalueof{/child count/#3}}
    \node [commit, anchor=east] (#2) at ($ (#3.west) + (-.5,\value{tmp}) $) {#1};
    \draw [commit line] (#2) edge (#3);
    \addtocounter{tmp}{1}
    \pgfkeys{/child count/#2/.initial=0}
    \pgfkeys{/child count/#3=\value{tmp}}
}

\def\branch #1 -> #2;{\ref [branch] "#1" : #1 -> #2;}
\def\oldbranch #1 -> #2;{\oldref [branch] "#1" : #1 -> #2;}

\def\addHEAD [#1] "#2" : #3 -> #4;{
    \begin{pgfonlayer}{HEAD}
        \node [#1, anchor=south] (#3) at (#4.south) {};
        \node [#1 label, anchor=south] (#3 label) at ($(#4.north)+(0,-3pt) $) {#2};
    \end{pgfonlayer}
}

\def\HEAD -> #1;{\addHEAD [attached HEAD] "HEAD" : HEAD -> #1;}
\def\oldHEAD -> #1;{\addHEAD [old attached HEAD] "HEAD" : old HEAD -> #1;}

\def\detachedHEAD -> #1;{\ref [HEAD] "HEAD" : HEAD -> #1;}
\def\olddetachedHEAD -> #1;{\oldref [HEAD] "HEAD" : HEAD -> #1;}

\pagestyle{empty}
\setlength{\parindent}{0pt}
